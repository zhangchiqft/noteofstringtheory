\documentclass[cn,device=normal,color=blue]{elegantnote}

\title{\bfseries Harmonic Analysis: \\ \large{on the $n$-Dimensional Lorentz Group and Its Application to Conformal Field Theory}}

\author{ 张\,驰\thanks{\texttt{email:zhangchi\_110@163.com}} }
\date{\today}
%\usepackage{amsmath}
\usepackage{tcolorbox}
\usepackage{listings}
\usepackage{arydshln}
\usepackage{dsfont}
%\usepackage{bbold}
\renewcommand\theequation{\arabic{section}.\arabic{equation}} 
\lstset{language=[LaTeX]{TeX},basicstyle=\footnotesize\ttfamily}


\definecolor[named]{DarkBlue}{HTML}{4f81bc}
\definecolor[named]{SemiDarkBlue}{HTML}{233e5f}
\definecolor[named]{LightBlue}{HTML}{dbe4f0}
\definecolor[named]{Nblue}{RGB}{58,58,144}
\definecolor[named]{darkred}{RGB}{121,0,38}

% %+++++++++++++++++++++++数学字体的设置++++++++++++++++++++++++++++++++++++++++%
\newcommand{\me}{\mathrm{e}}  % for math e
\newcommand{\mi}{\mathrm{i}} % for math i
\newcommand{\dif}{\mathrm{d}} %for differential operator d
% \newcommand{\cvec}[1]{\!\vec{\,#1}}
% \newcommand{\Ptimes}{\,\overset{\otimes }{,}\,}
% \DeclareSymbolFont{lettersA}{U}{txmia}{m}{it}
%  \DeclareMathSymbol{\piup}{\mathord}{lettersA}{25}
%  \DeclareMathSymbol{\muup}{\mathord}{lettersA}{22}
%  \DeclareMathSymbol{\deltaup}{\mathord}{lettersA}{14}


\begin{document}
{\color{ecolor}{\maketitle}}
% logo
\maketitle

\newpage

\section{群论基础}

\subsection{\texorpdfstring{$O(2h{+}1,1)\,$}{O(2h+1,1)}群和它的\,Lie\,代数}

记$\,N = 2h{+}1$. $h\,$是半整数, 接下来主要考虑$\,h>1\,$的情况.

$O(2h{+}1,1)\,$群是保持{\it{实}}\,$(N{+}1)$-维矢量空间中的如下内积
\begin{gather}
    \xi^{2}:= \xi\eta\xi =\xi_{1}^{2}+\cdots +\xi_{N}^{2}-\xi_{0}^{2}\quad  \label{1.1} 
\end{gather}
不变的线性变换构成的集合. ($\eta\,$是对角矩阵, 其中$\,\eta_{11}=\cdots=\eta_{NN}=-\eta_{00}=1$.) 
这个群由满足
\begin{equation}
    g^{\mathrm{T}}\eta g =\eta 
\end{equation}
的$\,(N{+}1)\times (N{+}1)\,$实矩阵$\,g\,$构成. 它的恒元分量$\,G=SO^{\uparrow}(N,1)\,$由满足\eqref{1.1}以及
\begin{equation}
    \operatorname{det}g=1,\quad g^{0}{}_{0}>1
\end{equation}
的那些$\,g\,$组成. 我们将主要处理包含空间反射的扩张群
\begin{equation}
    G_{\text{ex}}=O^{\uparrow}(2h{+}1,1)=\{g\in O(2h{+}1,1)\,\vert\,g^{0}{}_{0}\geq 1\}.
\end{equation}

$G\,$的\,Lie\,代数$\,\mathfrak{g}\,$由满足
\begin{equation}
    X^{\mathrm{T}}\eta+\eta X=0 \label{1.5}
\end{equation}
的$\,(N{+}1)\times (N{+}1)\,$实矩阵$\,X\,$构成. 我们可以选择$\,\mathfrak{g}\,$中的一组基$\,X_{AB}=-X_{BA}$\,($\,A,B=0,1,\ldots,N\,$)使其满足对易关系
\begin{equation}
    [X_{AB},X_{CD}]=\eta_{AC}X_{BD}+\eta_{BD}X_{AC}-\eta_{AD}X_{BC}-\eta_{BC}X_{AD} \:.\label{1.6}
\end{equation}
(注意物理中使用的$\,J_{AB}\,$与$\,X_{AB}\,$的关系是$\,J_{AB}=\mi X_{AB}$.) $X_{AB}\,$的一个矩阵实现是
\begin{equation}
    \bigl(X_{AB}\bigr)^{C}_{D}=\eta_{AD}\delta_{B}^{C}-\eta_{BD}\delta_{A}^{C}\:. \label{1.7}
\end{equation}

\subsection{子代数和分解}

下表给出了群$\,G\,$的几个重要子群:
\begin{table}[h]
\centering
        \begin{tabular}{ c c c }
         子群 & 特性 & 生成元 \\ \hline
         $K=SO(2h{+}1)$ & 最大紧子群 & $X_{ab}\:\text{其中  } a,b=1,\ldots,2h{+}1$ \\  
         $A=SO(1,1)$ & 1-维非紧群(``伸缩变换'') & $X_{2h{+}1,0}=D$ \\
         $M=SO(2h)$  & $A\,$在$\,K\,$中的中心($mam^{-1}=a$)\, & $X_{\mu\nu}\:\text{其中  } \mu,\nu=1,\ldots,2h$ \\  
         $N$ & 幂零, 阿贝尔 (特殊共形变换) & $C_{\mu}=X_{\mu,0}-X_{\mu,2h+1}$ \\
         $\tilde{N}$ & 幂零, 阿贝尔 (平移) & $T_{\mu}=X_{\mu,0}+X_{\mu,2h+1}$ \\
         $H=H_{AM}=AH_{M}$ & $(\lfloor h\rfloor{+}1)$-维阿贝尔群 (嘉当子群) & $D, X_{12},\ldots,X_{2\lfloor h\rfloor-1,2\lfloor h\rfloor}$ 
        \end{tabular}
        \caption{群$\,G=SO^{\uparrow}\,$的几个重要子群.}
\label{tab:1}
\end{table}

根据\eqref{1.6}, 生成元$\,D$, $C_{\mu}\,$和$\,T_{\mu}\,$满足对易关系:
\begin{align*}
    [D,C_{\mu}]=-C_{\mu}\:, \quad [D,T_{\mu}]=T_{\mu}\:,\quad \tfrac{1}{2}[T_{\mu},C_{\nu}]=D\delta_{\mu\nu}-X_{\mu\nu}; \nonumber \\
    [X_{\lambda\mu},T_{\nu}]=\delta_{\lambda\nu}T_{\mu}-\delta_{\mu\nu}T_{\lambda} \:, \quad 
    [X_{\lambda\mu},C_{\nu}]=\delta_{\lambda\nu}C_{\mu}-\delta_{\mu\nu}C_{\lambda} \:. \label{1.6'} \tag{1.6$'$}
\end{align*}
物理动量算符由$\,\mi T_{\mu}\,$的厄密表示给出.

令\,$M'$\,是$\,A\,$在$\,K\,$中的正规化子, 即对所有的$\,a\in A\,$使得$\,m'am'^{-1}\in A\,$的$\,m'\in K\,$的集合. 很容易看到$\,M'\,$同构于$\,O(2h)\,$且$\,M\,$是$\,M'\,$的不变子群. 有限群
\begin{equation}
    W=W(G,A)=M'/M
\end{equation}
被称为$\,(G,A)\,$对的\,Weyl\,群(\,或者限制\,Weyl\,群). 它有两个元素$\,W=\{1,w\}$. 幂零子群$\,N\,$和$\,\tilde{N}\,$在\,Weyl\,变换下互为共轭
\begin{equation}
    wNw^{-1}=\tilde{N} \:. \label{1.9}
\end{equation}
如有需要, 我们对$\,w\,$选择如下表示
\begin{equation}
    w=\exp\bigl(\pi X_{2h,2h+1}\bigr) \label{1.10}
\end{equation}
即在$\,(2h,2h{+}1)$-平面中旋转$\,\pi$. 在下文中, 我们将把$\,K,N,A,M,\tilde{N}$\,中的元素记为$\,k,n,a,m,\tilde{n}$.
\begin{tcolorbox}
    \begin{note}
        $\,K\,$中$\,X_{\mu\nu}\,$以外的生成元是$\,X_{\mu,2h+1}$, 它与$\,D=X_{2h+1,0}\,$的关系是
        \begin{equation*}
         \operatorname{Ad}({X_{\mu,2h+1}})D:=  [X_{\mu,2h+1},D]=-X_{\mu,0} \quad 
         \operatorname{Ad}^{2}(X_{\mu,2h+1})D=-D 
        \end{equation*}
    那么根据\,Baker–Campbell–Hausdorff\,公式
    \begin{align*}
        \exp(t X_{\mu,2h+1}) \exp(a D )\exp(-t X_{\mu,2h+1})
        &=\exp\biggl(a \sum_{n}\frac{t^{n}\operatorname{Ad}^{n}(X_{\mu,2h+1})D}{n!}\biggr) \\
        &=\exp\Bigl( a\cos(t)D-a\sin(t)X_{\mu,2h{+}1}\Bigr)
    \end{align*}
    由此得出$\,A\,$在$\,K\,$中的中心化子至多再含有\,$\exp(n\pi  X_{\mu,2h{+}1})$, 这个元素加上$\,M\,$构成了$\,O(2h)$. \eqref{1.9}式说明了: 平移作用在原点上但保持无限远不动, 而特殊共形变换则正相反.
    \end{note}
\end{tcolorbox}

\paragraph*{Iwasana\,分解:} $G\,$的每个群元可以表示为
\begin{equation}
    g=kna \:, \label{1.11}
\end{equation}
或
\begin{equation}
    g=\tilde{n}ak \label{1.12}
\end{equation}
(\eqref{1.11}和\eqref{1.12}中的$\,k\,$和$\,a\,$因子一般不同). 两种分解版本中的顺序十分重要, 在下面我们将使用\eqref{1.11}.

\paragraph*{Bruhat\,分解:} $G\,$的几乎所有群元$\,g\in G$\,(即那些不属于低维子流形$\,wNAM\,$的群元)都可以唯一地写成
\begin{equation}
    g=\tilde{n}nam \:. \label{1.13}
\end{equation}

\begin{remark}
对于奇数或偶数的$\,N$, $SO^{\uparrow}(N,1)\,$有一个很大的不同. 对于偶数的$\,N\,$ 除了表:\ref{tab:1}所列的嘉当子群外, 还有另外一个紧嘉当子群, 生成元是$\,X_{12},X_{34},\ldots,X_{N-1,N}$. 由此得出, 当$\,N\,$为偶数时, 椭圆元(本征值模为\,1\,的矩阵)集合的维数与$\,G\,$相同. 这使得进对于偶数的$\,N$, $G\,$的幺正表示是离散序列.
\end{remark}

\subsection{紧化欧几里得空间作为$\,G\,$的齐次空间}

出现在\,Bruhat\,分解中的``抛物子群''$\,NAM\,$在群$\,G\,$的诱导表示的\,Harish-Chandra\,构造中起到了特殊作用. 值得注意的是齐次空间
\begin{equation}
    G/NAM \approx K/M \approx S^{2h} \label{1.14}
\end{equation}
(同构于$\,2h{+}1\,$维中的单位球面$\,S^{2h}\,$)与物理有关. 它可以视为是\,$2h$\,-维欧几里得空间$\,\mathbb{X}=\mathbb{R}^{2h}$\,的共形紧化. ($S^{2h}\,$是通过给$\,\mathbb{X}\,$加上无穷远点$\,\infty\,$获得的.)

特别的, $G\,$在矢量空间$\,\mathbb{X}\,$上有一个定义良好的定域作用, 这个作用可以等同为右陪集$\,\tilde{n}NMA\,$的流形. 
由于$\,g\in G\backslash wNAm$的\,Bruhat\,分解\eqref{1.13}的唯一性, 陪集$\,x\in\mathbb{X}\,$与群元$\,\tilde{n}\in\tilde{N}\,$存在对应使得$\,x=\tilde{n}NMA$. 由于这个原因, 我们将$\,\tilde{N}\,$中与$\,x(\in\mathbb{X})\,$对应的元素记为$\,\tilde{n}_{x}$. 我们以
\begin{equation}
    \tilde{n}_{x_{1}}\tilde{n}_{x_{2}}=\tilde{n}_{x_{1}+x_{2}} \tag{1.15a}\label{1.15a}
\end{equation}
方式赋予$\,\mathbb{X}\,$以实矢量空间的结构. $G\,$的抛物子群$\,\tilde{N}MA\,$是$\,\mathbb{X}\,$的自同构群. 特别的, 
\begin{equation}
    \tilde{n}_{x}y=x+y \:. \tag{1.15b} \label{1.15b}
\end{equation}
零矢量$\,x=0\,$的稳定子群是$\,G\,$的抛物子群$\,NAM$. 齐次空间\eqref{1.14}同构于$\,\mathbb{R}^{2h+1,1}\,$中类光射线的集合:
\begin{equation*}
    G/NAM\approx \mathbb{K}_{+}/\mathbb{R}_{+} \tag{1.14$'$} \label{1.14'}
\end{equation*}
其中$\,\mathbb{R}_{+}\,$是正实数的乘法群, 而
\begin{equation*}
    \mathbb{K}_{+}:= \mathbb{K}^{2h+1,1}_{+}(\approx G/NM) = 
    \{ \xi\in \mathbb{R}^{2h+2};\:\xi_{0}>0,\:\xi\eta\xi=0\}
\end{equation*}
$x\in \mathbb{X}\,$用$\,\xi\,$表示是
\begin{equation}
    x_{\mu}=\frac{\xi_{\mu}}{\xi_{0}+\xi_{2h+1}} \:, \quad 
    \xi_{M}:=[\xi_{0}:\xi_{\mu}:\xi_{2h+1}]=[1+x^{2}:2x_{\mu}:1-x^{2}]\:. \tag{1.16}\label{1.16}
\end{equation}
\setcounter{equation}{16}
$g\,$在齐次空间\eqref{1.14'}上的左传递生成了$\,G\,$在$\,\mathbb{K}_{+}\,$的自然作用$\,g:$ $\xi\to g\xi$, 这给出了$\,x\,$的变换规则. 我们推出$\,\mathbb{X}\,$的自同构群是抛物子群$\,\tilde{N}AM\,$(与$\,NAM\,$共轭). $\tilde{N}\,$的作用已经在\eqref{1.15b}中给出. 伸缩$\,a\in A$\,与旋转$\,m\in M\,$在$\,\mathbb{X}\,$上的作用是
    \begin{subequations}\label{1.17}
        \begin{align}
            ax&=\lvert a\rvert x \:, \\
        (mx)_{\mu}&= m_{\mu\nu}x_{\nu} \:.
        \end{align}
    \end{subequations}
由于特殊线性变换总可以把$\,\mathbb{X}\,$中的任意一点送到无穷远, 我们只讨论它的无限小版本. 对于$\,N\,$中单位元附近的元素$\,n_{\epsilon}$, 我们有
\begin{equation}
    n_{\epsilon}x : x\to x'(\epsilon),\quad 
     x'_{\mu}=x_{\mu}+(x^{2}\delta_{\mu\nu}-2x_{\mu}x_{\nu})\epsilon_{\nu}+O(\epsilon^{2}) \:. \label{1.18}
\end{equation}

\subsection{$G\,$的各种子群的矩阵实现. Bruhat\,分解的构造}

矩阵$\,\tilde{n}\,$和$\,n\,$可以分别被相应的\,$2h$-维矢量$\,x(\in \mathbb{X})\,$和$\,b\,$实现. 
利用生成元的矩阵实现\eqref{1.7}, 我们获得矩阵$\,x_{\mu}T_{\mu}\,$和 $b_{\mu}C_{\mu}\,$的如下表示:
\begin{equation}
    x_{\mu}T_{\mu}= \left(
        \begin{array}{@{}c:c@{}}
            \mathbf{0}_{2h\times 2h} & \begin{matrix}
                -x^{\mathrm{T}} & x^{\mathrm{T}} 
            \end{matrix} \\ \hdashline[4pt/4pt]
            \begin{matrix}
                x \\ x
            \end{matrix} &
            \mathbf{0}_{2\times 2}
        \end{array}
    \right) \tag{1.19a} \label{1.19a}
\end{equation}
其中$\,x=(x_{1},\ldots,x_{2h})\,$是行矢量; 类似的,$\mathds{0}$
\begin{equation}
    b_{\mu}C_{\mu}= \left(
        \begin{array}{@{}c:c@{}}
            \mathbf{0}_{2h\times 2h} & \begin{matrix}
                -b^{\mathrm{T}} & b^{\mathrm{T}} 
            \end{matrix} \\ \hdashline[4pt/4pt]
            \begin{matrix}
                -b \\ b
            \end{matrix} &
            \mathbf{0}_{2\times 2} 
        \end{array}
    \right) \tag{1.19b} \label{1.19b}
\end{equation}
由此可以得到
\begin{equation}
    \tilde{n}_{x}=\exp(x_{\mu}T_{\mu}) =
    \begin{pmatrix}
        \mathds{1}_{2h\times 2h} & -x^{\mathrm{T}} &  x^{\mathrm{T}} \\
        x  & 1-\tfrac{1}{2}x^{2} & \tfrac{1}{2}x^{2} \\
        x & -\tfrac{1}{2}x^{2} & 1+\tfrac{1}{2}x^{2}
    \end{pmatrix}  \tag{1.20a} \label{1.20a}
\end{equation}
和
\begin{equation}
    n_{b}=\exp(b_{\mu}C_{\mu}) =
    \begin{pmatrix}
        \mathds{1}_{2h\times 2h} & b^{\mathrm{T}} &  b^{\mathrm{T}} \\
        -b  & 1-\tfrac{1}{2}b^{2} & -\tfrac{1}{2}b^{2} \\
        b & \tfrac{1}{2}b^{2} & 1+\tfrac{1}{2}b^{2}
    \end{pmatrix}  \tag{1.20b} \label{1.20b} \:.
\end{equation}
\setcounter{equation}{20}

类似地, 伸缩变换$\,a\,$的矩阵表示是
\begin{equation}
    a=\exp(\alpha D) =
    \begin{pmatrix}
        \mathds{1}_{2h\times 2h} & 0&  0 \\
        0  &\cosh \alpha & \sinh \alpha \\
        0 & \sinh \alpha & \cosh \alpha
    \end{pmatrix}  
\end{equation}
\begin{note}
    在这个变换下, $(\xi_{2h+1},\xi_{0})\to ( \xi_{2h+1}\cosh \alpha+\xi_{0}\sinh \alpha , \xi_{2h+1}\sinh \alpha+ \xi_{0}\cosh \alpha)$, 所以$\,\xi_{2h+1}+\xi_{0}\to \me^{\alpha}(\xi_{2h+1}+\xi_{0})$, 那么$\,a x= \me^{-\alpha}x$.
\end{note}
同样有
\begin{align}
    m&= \begin{pmatrix}
        m_{\mu\nu} & 0 & 0 \\
        0 & 1 & 0 \\
        0 & 0 & 1 \\
    \end{pmatrix} \:, \qquad (m_{\mu\nu})\in SO(2h) \:, \label{1.22} \\
    k&= \begin{pmatrix}
        k_{\mu\nu} & k_{\mu,2h+1} & 0 \\
        k_{2h+1,\nu} & k_{2h+1,2h+1} & 0 \\
        0 & 0 & 1 \\
    \end{pmatrix} \:, \qquad (k_{ab})\in SO(2h+1) \:. \label{1.22}
\end{align}
将矩阵\eqref{1.20a}---\eqref{1.22}乘起来就得到了群元$\,g\,$形如\eqref{1.13}的表达式
\begin{align}
    g&=\tilde{n}_{x}n_{b}am =
    \begin{pmatrix}
        \mathds{1}_{2h\times 2h} & -x^{\mathrm{T}} &  x^{\mathrm{T}} \\
        x  & 1-\tfrac{1}{2}x^{2} & \tfrac{1}{2}x^{2} \\
        x & -\tfrac{1}{2}x^{2} & 1+\tfrac{1}{2}x^{2}
    \end{pmatrix}
    \begin{pmatrix}
        m^{\mu}{}_{\nu} & \me^{\alpha}b^{\mu} & \me^{\alpha}b^{\mu} \\
        -b_{\sigma}m^{\sigma}{}_{\nu} & \frac{1}{2}\bigl(\me^{\alpha}(1-b^{2})+\me^{-\alpha}\bigr)&
        \frac{1}{2}\bigl(\me^{\alpha}(1-b^{2})-\me^{-\alpha}\bigr) \\
        b_{\sigma}m^{\sigma}{}_{\nu} &\frac{1}{2}\bigl(\me^{\alpha}(1+b^{2})-\me^{-\alpha}\bigr)&
        \frac{1}{2}\bigl(\me^{\alpha}(1+b^{2})+\me^{-\alpha}\bigr)
    \end{pmatrix} \nonumber \vspace{10pt} \\ 
    &= \begin{pmatrix}
        m^{\mu}{}_{\nu}+2x^{\mu}b_{\sigma}m^{\sigma}{}_{\nu} & \me^{\alpha} b^{\mu}+ (\me^{\alpha}b^{2}-\me^{-\alpha})x^{\mu} &
        \me^{\alpha}b^{\mu}+ (\me^{\alpha}b^{2}+\me^{-\alpha})x^{\mu}  \\
        x_{\sigma}m^{\sigma}{}_{\nu}+b_{\sigma}m^{\sigma}{}_{\nu}(x^{2}{-}1) &
        \me^{\alpha} x_{\rho}b_{\rho} +\frac{\bigl( \me^{\alpha}+(\me^{\alpha}b^{2}-\me^{-\alpha})(x^{2}-1)\bigr)}{2} &
        \me^{\alpha} x_{\rho}b_{\rho} +\frac{\bigl( \me^{\alpha}+(\me^{\alpha}b^{2}+\me^{-\alpha})(x^{2}-1)\bigr)}{2} \\
        x_{\sigma}m^{\sigma}{}_{\nu}+b_{\sigma}m^{\sigma}{}_{\nu}(x^{2}{+}1) &
        \me^{\alpha} x_{\rho}b_{\rho} +\frac{\bigl( \me^{\alpha}+(\me^{\alpha}b^{2}-\me^{-\alpha})(x^{2}+1)\bigr)}{2} &
        \me^{\alpha} x_{\rho}b_{\rho} +\frac{\bigl( \me^{\alpha}+(\me^{\alpha}b^{2}+\me^{-\alpha})(x^{2}+1)\bigr)}{2}
    \end{pmatrix} \label{1.24}
\end{align}
矩阵$\,g\in G\,$可以写成形如\eqref{1.24}的条件是
\begin{equation}
    d(g):=\tfrac{1}{2}\bigl(g^{2h+1}{}_{2h+1}-g^{2h+1}{}_{0}-g^{0}{}_{2h+1}+g^{0}{}_{0}\bigr) =\me^{-\alpha} >0 \:. \label{1.25}
\end{equation}
注意, $d(g)\,$对形如$\,wnam\,$的$\,g\,$为零.

利用\eqref{1.24}可以得到
\begin{subequations}
    \begin{align}
        x^{\mu}&=\frac{1}{2d(g)}\Bigl(g^{\mu}{}_{0}-g^{\mu}{}_{2h+1}\Bigr) \:, \\
        2b^{\mu}&=\bigl(g^{0}{}_{0}-g^{2h+1}{}_{0}\bigr)g^{\mu}{}_{2h+1}+
        \bigl(g^{2h+1}{}_{2h+1}-g^{0}{}_{2h+1}\bigr)g^{\mu}{}_{0} \\
        m^{\mu}{}_{\nu}&=g^{\mu}{}_{\nu}-\frac{\bigl(g^{\mu}{}_{0}-g^{\mu}{}_{2h+1}\bigr)\bigl(g^{0}{}_{\nu}-g^{2h+1}{}_{\nu}\bigr)}{2d(g)}
    \end{align} \label{1.26}
\end{subequations}

通过
\begin{equation}
    g\tilde{n}_{x} =\tilde{n}_{x'} n(g,x)a(g,x)m(g,x) \:. \label{1.27}
\end{equation}
我们定义变换$\,x\xrightarrow[]{g}x'$. 这给出了在$\,\mathbb{X}\,$的自同构群$\,\tilde{N}AM\,$下的变换规则\eqref{1.15b}, \eqref{1.17}, 以及特殊共形变换规则
\begin{equation}
    x'^{\mu} = \frac{x^{\mu}+x^{2}b^{\mu}}{1+2b_{\nu}x^{\nu}+b^{2}x^{2}} \:. \label{1.28}
\end{equation}
注意到\eqref{1.28}中的分母正是\eqref{1.25}中定义的$\,d(n_{b}\tilde{n}_{x})$. 特殊共形变换可以表示为平移和$\,(2h{+}1)\,$-轴的反射$\,R\,$($\,R\,$等于\eqref{1.10}定义的\,Weyl\,变换$\,w\,$再加上反射$\,\xi_{i}\to-\xi_{i}$, $i=1,\ldots,2h-1$.) 利用$\,R\tilde{n}_{x}\,$的分解\eqref{1.27}, 我们得到在{\it 共形反演}下的变换:
\begin{equation}
    Rx^{\mu}=-\frac{x^{\mu}}{x^{2}} \: ;\qquad 
    wx^{\mu} = \frac{\theta x^{\mu}}{x^{2}} \:,
\end{equation}
其中$\,\theta\,$是$\,2h\,$-轴的反射, 很容易证明$\,n_{b}=R\tilde{n}_{b}R^{-1}$.

我们回到\eqref{1.27}. 通过
\begin{equation}
    g^{-1}\tilde{n}_{x}=\tilde{n}_{x'}p(x,g)^{-1}\:, \qquad x'=g^{-1}x \:, \tag{1.27a}\label{1.27a}
\end{equation}
我们定义$\,p(x,g)\in MAN$. 那么$\,p(x,g^{-1})^{-1}=n(g,x)a(g,x)m(g,x)$. 从\eqref{1.27a}中可以推出上闭链条件
\begin{equation}
    p(x,g_{1}g_{2})=p(x,g_{1})p(g_{1}^{-1}x,g_{2}) \:. \label{1.27b} \tag{1.27b}
\end{equation}
\begin{tcolorbox}
   \begin{equation*}
    \tilde{n}_{g_{2}^{-1}g_{1}^{-1}x}p(x,g_{1}g_{2})^{-1}=g_{2}^{-1}g_{1}^{-1}\tilde{n}_{x}
    =g_{2}^{-1}\tilde{n}_{g_{1}^{-1}x}p(x,g_{1})^{-1}=\tilde{n}_{g_{2}^{-1}g_{1}^{-1}x}p(g_{1}^{-1}x,g_{2})^{-1}p(x,g_{1})^{-1}
   \end{equation*} 
\end{tcolorbox}
\noindent 从定义可以立即得出如下的特殊情况
\begin{equation}
    p(x,\tilde{n})=1 \qquad \text{对于}\quad \tilde{n}\in\tilde{N}\:; \qquad
    p(x,ma)=ma \qquad \text{对于}\quad ma\in MA \:. \label{1.27c} \tag{1.27b}
\end{equation}
对于
\begin{equation}
    p_{x}:=p(x,w) \:. \label{1.27d} \tag{1.27d}
\end{equation}
方程(\ref{1.24})给出显式公式
\begin{equation}
    p_{x}:=m_{x}a_{x}n_{\theta x} \qquad \text{其中} \:
    m_{x}=\biggl(-\mathds{1}+2\frac{xx^{\mathrm{T}}}{x^{2}}\biggr)I_{s}\:,\quad \lvert a(x)\rvert =x^{2} \:. 
    \label{1.27e} \tag{1.27e}
\end{equation}
其中$\,x,x^{\mathrm{T}}\,$分别列矢量和行矢量$\,(x^{1},\ldots,x^{2h})$. 可以立即得到如下恒等式
\begin{equation}
m_{x+y}m_{wy}=m_{x}m_{wx+wy}\:; \qquad a_{x+y}a_{wy}=a_{x}a_{wx+wy} \label{1.27f} \tag{1.27f}
\end{equation}
和
\begin{equation}
p(x,n_{-Ry})=h_{x}h_{wx-wy}n_{Ru} =n_{R(y-x)}h_{x}h_{wx-wy} \label{1.27g} \tag{1.27g}
\end{equation}
其中$\,h_{x}=m_{x}a_{x}\,$且$\,u=y+w(wx-wy)$.
\begin{note}
    HW: 证明\eqref{1.27e}---\eqref{1.27g}. $I_{s}\,$是什么?
\end{note}

\subsection{Bruhat\,分解与\,Iwasawa\,分解之间的关系}

\end{document}