\usepackage{amsmath}                      % AMSLaTeX宏包,用来排出更加漂亮的公式
%\usepackage{CJKutf8}
%\usepackage{amsfonts}                     % AMS提供的数学符号的字库
%\usepackage{amssymb}                      % 数学符号生成命令
%\usepackage{unicode-math}
\usepackage{amsthm}                       % 数学定理环境
\usepackage[lcgreekalpha]{stix}
%\usepackage{upgreek}
\usepackage{anyfontsize}
\usepackage{wasysym}                      % 支持直立的积分号
\usepackage{indentfirst}                  % 首行缩进宏包
\usepackage{geometry}                     % 设置页边距
%\usepackage{eufrak}                       % 引入德文字体
 \usepackage{hypbmsec}                     % 用来控制书签中标题显示内容
 \usepackage{graphicx}                     % 支持插图处理
 %\usepackage{epstopdf}                     % eps图片转换成pdf
 \usepackage{caption}                     % 浮动体标题的格式控制
 \usepackage{mathrsfs}                    %支持花写字母
 \usepackage{tikz}
 \usepackage{bm}
% \usepackage{xeCJK}
% \usepackage{upgreek}
 %\usepackage{stix}
 \usetikzlibrary{arrows}  
% \usepackage[CJKbookmarks=true,
%             unicode,
%             hyperfootnotes=true,
%             bookmarks=true,
%             colorlinks,
%             citecolor=blue]{hyperref}     % 中文书签

\geometry{left=2.5cm,right=2.5cm, top=3.0cm, bottom=3.0cm}
\usepackage{dsfont}   %空心数字
%\usepackage{shuffle}


\newcommand{\song}{\CJKfamily{song}}    % 宋体   (Windows自带simsun.ttf)
\newcommand{\fs}{\CJKfamily{fs}}        % 仿宋体 (Windows自带simfs.ttf)
\newcommand{\kai}{\CJKfamily{gkai}}      % 楷体   (Windows自带simkai.ttf)
\newcommand{\hei}{\CJKfamily{hei}}      % 黑体   (Windows自带simhei.ttf)

%+++++++++++++++++++++++数学字体的设置++++++++++++++++++++++++++++++++++++++++%
\newcommand{\me}{\mathrm{e}}  % for math e
\newcommand{\mi}{\mathrm{i}} % for math i
\newcommand{\dif}{\mathrm{d}} %for differential operator d
\newcommand{\Li}{\operatorname{Li}} %for Polylog Li
\newcommand{\Reg}{\operatorname{Reg}} %for regularized limits
%\newcommand{\det}{\operatorname{det}}
\newcommand{\Pf}{\operatorname{Pf}}                % for mathematical Pfaffian
\newcommand{\cvec}[1]{\!\vec{\,#1}}               %居中的矢量箭头
\newcommand{\Ptimes}{\,\overset{\otimes }{,}\,}   %张量Poisson括号
% \DeclareSymbolFont{lettersA}{U}{txmia}{m}{it}
% \DeclareMathSymbol{\piup}{\mathord}{lettersA}{25}
%  \DeclareMathSymbol{\muup}{\mathord}{lettersA}{22}
%  \DeclareMathSymbol{\deltaup}{\mathord}{lettersA}{14}
%  \newcommand{\uppi}{\piup}


%\renewcommand{\captionlabeldelim}{\ }%去掉图标签后面的冒号

\setcounter{section}{0}%更改chapter的计数器值
\numberwithin{equation}{section}%公式计数器从属于节计数器
\numberwithin{figure}{section}%图计数器从属于节计数器
\pdfmapfile{=pdftex.map}

