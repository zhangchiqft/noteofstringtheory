
\usepackage{amsmath}                      % AMSLaTeX宏包,用来排出更加漂亮的公式
\usepackage{CJKutf8}                          % 中文排版
\usepackage{amsfonts}                     % AMS提供的数学符号的字库
\usepackage{amssymb}                      % 数学符号生成命令
\usepackage{amsthm}                       % 数学定理环境
 \usepackage[]{xcolor}    %颜色
%\usepackage{newpxmath}
\usepackage{mathpazo}
\usepackage{wasysym}                      % 支持直立的积分号
\usepackage{indentfirst}                  % 首行缩进宏包
\usepackage{geometry}                     % 设置页边距
%\usepackage{eufrak}                       % 引入德文字体
 \usepackage{hypbmsec}                     % 用来控制书签中标题显示内容
 \usepackage{graphicx}                     % 支持插图处理
 \usepackage{epstopdf}                     % eps图片转换成pdf
 \usepackage{caption}                     % 浮动体标题的格式控制
 \usepackage{bm}                           %斜黑体 
 \usepackage{mathrsfs}                   % 花体英文
 \usepackage{tcolorbox}                  % 盒子
 \usepackage{tensor}
 \usepackage{dsfont}                     % 空心字母  
 %\tcbuselibrary{breakable}
 %\usepakage{galois}                 %复合函数符号
 \usepackage{framed}

 %\usepackage{colorboxed}
 \definecolor{foo}{RGB}{173,34,48}
 \usepackage[CJKbookmarks=true,
            unicode,
            hyperfootnotes=true,
            bookmarks=true,
            colorlinks=true,
            linkcolor=foo,
            citecolor=blue]{hyperref}     % 中文书签
\usepackage{dsfont}   %空心数字