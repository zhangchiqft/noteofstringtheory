
\geometry{left=2.0cm,right=2.0cm, top=2.5cm, bottom=2.5cm}
 \DeclareFontEncoding{LS1}{}{}
\DeclareFontSubstitution{LS1}{stix}{m}{n}
\DeclareSymbolFont{symbols2}{LS1}{stixfrak}{m}{n}
\DeclareMathSymbol{\typecolon}{\mathbin}{symbols2}{"25}
\newcommand{\song}{\CJKfamily{song}}    % 宋体   (Windows自带simsun.ttf)
\newcommand{\fs}{\CJKfamily{fs}}        % 仿宋体 (Windows自带simfs.ttf)
\newcommand{\kai}{\CJKfamily{gkai}}      % 楷体   (Windows自带simkai.ttf)
\newcommand{\hei}{\CJKfamily{hei}}      % 黑体   (Windows自带simhei.ttf)

%+++++++++++++++++++++++数学字体的设置++++++++++++++++++++++++++++++++++++++++%
\newcommand{\me}{\mathrm{e}}  % for math e
\newcommand{\mi}{\mathrm{i}} % for math i
\newcommand{\dif}{\mathrm{d}} %for differential operator d
\newcommand{\cvec}[1]{\!\vec{\,#1}}
\newcommand{\Ptimes}{\,\overset{\otimes }{,}\,}
\DeclareSymbolFont{lettersA}{U}{txmia}{m}{it}
 \DeclareMathSymbol{\piup}{\mathord}{lettersA}{25}
 \DeclareMathSymbol{\muup}{\mathord}{lettersA}{22}
 \DeclareMathSymbol{\deltaup}{\mathord}{lettersA}{14}
 \newcommand{\uppi}{\piup}


%\renewcommand{\captionlabeldelim}{\ }%去掉图标签后面的冒号

\setcounter{section}{0}%更改chapter的计数器值
%\numberwithin{equation}{chapter}%公式计数器从属于节计数器
\numberwithin{equation}{section}%公式计数器从属于节计数器
\numberwithin{figure}{section}%图计数器从属于节计数器
\pdfmapfile{=pdftex.map}
\setcounter{chapter}{9}

%-------------------- 用于中文段落缩进和正文版式 ------------------%

\setlength{\hoffset}{0cm}
\setlength{\voffset}{0cm}
\setlength{\parindent}{2em}                 % 首行两个汉字的缩进量
\setlength{\parskip}{3pt plus1pt minus1pt}  % 段落之间的竖直距离
\renewcommand{\baselinestretch}{1.2}        % 定义行距
\setlength{\abovecaptionskip}{5pt}
\setlength{\belowcaptionskip}{5pt}
\setlength{\abovedisplayskip}{8.5pt plus 3pt minus 4pt}
\setlength{\belowdisplayskip}{8.5pt plus 3pt minus 4pt}
\setlength{\abovedisplayshortskip}{20pt plus 2pt}
\setlength{\belowdisplayshortskip}{4pt plus 2pt minus 2pt}
%------------------------------------------------------------------%

